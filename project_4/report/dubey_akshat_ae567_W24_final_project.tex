%%%%%%%%%%%%%%%%%%%%%%%%%%%%%%%%%%%%%%%%%%%%%%%%%%%%%%%%%%%%%%%%%%%%%%%%%%%%%%%%
%2345678901234567890123456789012345678901234567890123456789012345678901234567890
%        1         2         3         4         5         6         7         8

\documentclass[letterpaper, 10 pt, conference]{ieeeconf}  % Comment this line out if you need a4paper

% \documentclass[a4paper, 10pt, conference]{ieeeconf}      % Use this line for a4 paper

\IEEEoverridecommandlockouts                              % This command is only needed if 
                                                          % you want to use the \thanks command

\overrideIEEEmargins                                      % Needed to meet printer requirements.

%In case you encounter the following error:
%Error 1010 The PDF file may be corrupt (unable to open PDF file) OR
%Error 1000 An error occurred while parsing a contents stream. Unable to analyze the PDF file.
%This is a known problem with pdfLaTeX conversion filter. The file cannot be opened with acrobat reader
%Please use one of the alternatives below to circumvent this error by uncommenting one or the other
%\pdfobjcompresslevel=0
%\pdfminorversion=4

% See the \addtolength command later in the file to balance the column lengths
% on the last page of the document

% The following packages can be found on http:\\www.ctan.org
%\usepackage{graphics} % for pdf, bitmapped graphics files
%\usepackage{epsfig} % for postscript graphics files
%\usepackage{mathptmx} % assumes new font selection scheme installed
%\usepackage{times} % assumes new font selection scheme installed
\usepackage{amsmath} % assumes amsmath package installed
\usepackage{amssymb}  % assumes amsmath package installed

\title{\LARGE \bf
Joint Parameter and State Estimation of a 2-DOF Gimbal
}

\author{Akshat Dubey$^{1}$% <-this % stops a space
\thanks{Prof. Alex Gorodetsky, University of Michigan}% <-this % stops a space
\thanks{$^{1}$Akshat Dubey is with Faculty of Aerospace Engineering,
        University of Michigan, 1320 Beal Ave, Ann Arbor, Michigan 48109 USA
        {\tt\small akshatdy@umich.edu}}%
}

\begin{document}

\maketitle
\thispagestyle{empty}
\pagestyle{empty}

%%%%%%%%%%%%%%%%%%%%%%%%%%%%%%%%%%%%%%%%%%%%%%%%%%%%%%%%%%%%%%%%%%%%%%%%%%%%%%%%

\section{BACKGROUND}
% The problem in consideration is one which I have personally experienced in my past work as an engineer building a terrestrial laser communication system. The key goal of the product was to point a laser very accurately, within 0.005 degrees to another laser terminal while remaining stable to enable robust communication. While manufacturing the system, there were always inconsistencies in the assembly and tolerance of parts, which led to variations in the system dynamics between units of the communication system. In addition, if not shipped with care, the dynamics could change between factory and deployment due to mechanical stresses. While the existing control scheme was able to stabilize the system, I was always curious about the possibility of estimating the system dynamics in real-time to improve the control performance.

For a laser communication system, it is essential for the system to point the laser accurately to another terminal.
This involves the use of a 2-DOF gimbal system to stabilize the laser so that it able to counteract the
mechanical disturbances and vibrations.
While manufacturing the system, there might be inconsistencies in the assembly and tolerance of parts, which can
lead to variations in the parameters that determine system dynamics between different units of the same product.
In addition, if not shipped with care, the dynamics could change between factory and deployment due to
mechanical stresses.
Hence, it is essential to estimate the system dynamics on the fly to be able to improve the control performance.
In addition, in the presence of noisy sensors, it is also important to estimate the state of the system accurately.

%%%%%%%%%%%%%%%%%%%%%%%%%%%%%%%%%%%%%%%%%%%%%%%%%%%%%%%%%%%%%%%%%%%%%%%%%%%%%%%%

\section{PROBLEM SETUP}
The dynamics for this system have been derived in \cite{c1} and are given by the following equations:
\begin{equation}
  \begin{aligned}
    \dot{x}_G = f_G(x_G, u_G, w_G) \\
    y_g = h_G(x_G, w_G, n_G)
  \end{aligned}
\end{equation}

where $x_G$ is the state $\in \mathbb{R}^{6}$, $u_G$ is the control input $\in \mathbb{R}^{2}$,
$w_G$ is the disturbance vector $\in \mathbb{R}^{8}$, $y_G$ is the output $\in \mathbb{R}^{4}$,
and $n_G$ is the vector of gyroscope noise $\in \mathbb{R}^{4}$. 

\begin{equation}
  \label{sample:states_and_shit}
  \begin{aligned}
    x_G & = [\eta, \epsilon, \Omega_{OBz}, \Omega_{IOy}, z_{OB}, z_{IO}]                                  \\
        & = [x_{G1}, x_{G2}, x_{G3}, x_{G4}, x_{G5}, x_{G6}]                                              \\
    u_G & = [T_{mO}, T_{mI}]                                                                              \\
        & = [u_{G1}, u_{G2}]                                                                              \\
    w_G & = [\theta, \phi, \dot{\psi}, \dot{\theta}, \dot{\phi}, \ddot{\psi}, \ddot{\theta}, \ddot{\phi}] \\
        & = [w_{G1}, w_{G2}, w_{G3}, w_{G4}, w_{G5}, w_{G6}, w_{G7}, w_{G8}]                              \\
    y_G & = [\eta, \epsilon, (\Omega_{OBz} + n_z), (\Omega_{IOy} + n_y)]                                  \\
        & = [y_{G1}, y_{G2}, y_{G3}, y_{G4}]                                                              \\
    n_G & = [0, 0, n_z, n_y]
  \end{aligned}
\end{equation}

$\eta$ corresponds to the gimbal pitch, or azimuth angle, that can be controlled.
$\epsilon$ corresponds to the gimbal yaw, or the elevation, that can be controlled.
$\Omega_{OBz}$ and $\Omega_{IOy}$ correspond to the rates of change of the gimbal pitch and yaw respectively.
$z_{OB}$, $z_{IO}$ correspond to the friction between the outer gimbal and the body and between the inner and outer gimbals respectively.
$\theta$, $\phi$, $\psi$ correspond to external disturbances applied in the pitch and yaw directions in the gimbal frame.
$T_{mO}$ and $T_{mI}$ correspond to the motor torques from the outer and inner gimbals to control the gimbal in response to these disturbances.
The state equation is written in the form $S_G \dot x_G = f_G(x_G, u_G, w_G)$, where $S_G$ is

\begin{equation}
  S_G =     \begin{bmatrix}
    1 & 0 & 0      & 0      & 0 & 0 \\
    0 & 1 & 0      & 0      & 0 & 0 \\
    0 & 0 & s_{g1} & s_{g2} & 0 & 0 \\
    0 & 0 & s_{g3} & 1      & 0 & 0 \\
    0 & 0 & 0      & 0      & 1 & 0 \\
    0 & 0 & 0      & 0      & 0 & 1 \\
  \end{bmatrix}
\end{equation}

where 

\begin{equation}
  \begin{aligned}
    s_{g1} & = 1 + 1.441 \cos(x_{G2})^2 - 1.451 \sin(x_{G2})^2 \\
    s_{g2} & = 0.0364 \cos(x_{G2}) + 0.0052 \sin(x_{G2})       \\
    s_{g3} & = -0.02345 \sin(x_{G2}) + 0.1274 \cos(x_{G2})
  \end{aligned}
\end{equation}

and the equation for the nonlinear dynamics is given by

\begin{equation}
  \label{nonlinear state space}
  \begin{split}
    S_G \dot x_G =
    \begin{bmatrix}
      x_{G3}                                                                          \\
      x_{G4}                                                                          \\
      -0.1 \hat{D}_{\text{O\_dynUnbz}} + 0.1 u_{G1} + 0.1 T_{\text{fr\_BO}} \hdots    \\ \hdots + 0.1 T_{\text{fr\_IO}}  + 0.1 D_{\text{O\_staticUnbz}} - \alpha_{\text{dist\_Oz}} \\
      -0.35 \hat{D}_{\text{I\_dynUnby}} + 0.35 T_{\text{fr\_OI}} + 0.35 u_{G2} \hdots \\ \hdots + 0.35 D_{\text{I\_staticUnby}} - \alpha_{\text{dist\_Iy}}                     \\
      x_{G3} - 4.42 \lvert x_{G3} \rvert x_{G5}                                       \\
      x_{G4} - 33.3 \lvert x_{G4} \rvert x_{G6}
    \end{bmatrix}
  \end{split}
\end{equation}

Equation \eqref{nonlinear state space} can be inverted to solve for $\dot{x_G}$.

$\Omega_{I_{ez}}$,  $\Omega_{dist_{Iy}}$, $\Omega_{Iex}$, $\hat{D}^O_{dynUnbz}$, $\hat{D}^I_{dynUnby}$,
$D^O_{staticUnbz}$, $D^I_{staticUnby}$ correspond to the nonlinear rotational dynamic terms between the
inner and outer gimbals, and are covered in the appendix.

The friction torques betwen the gimbals are given by

\begin{equation}
  \begin{aligned}
    T_{\text{fr\_OI}} & = 90 \cdot x_{G6} \quad \text{(inner and outer gimbal)}       \\
    T_{\text{fr\_IO}} & = 90 \cdot x_{G6} \quad \text{(inner and outer gimbal)}       \\
    T_{\text{fr\_BO}} & = 80 \cdot x_{G5} \quad \text{(gimbal base and outer gimbal)}
  \end{aligned}
\end{equation}

These friction torques will be the {\bf unknown parameters} in the problem setup

The equation for the output is given by

\begin{equation}
  \begin{split}
    y = 
    \begin{bmatrix}
      x_{G1}                                                           \\
      x_{G2}                                                           \\
      x_{G3} - w_{G4}sin(w_{G2}) + w_{G3}cos(w_{G2})cos(w_{G1}) \hdots \\ \hdots + n_z                                \\
      x_{G4} - w_{G4}sin(w_{G1}) + w_{G4}cos(w_{G1})cos(w_{G2}) \hdots \\ \hdots + w_{G3}cos(x_{G1})sin(w_{G2}) + n_y \\
    \end{bmatrix}
  \end{split}
\end{equation}

This outputs $y_{G1}$ and $y_{G2}$ will have a noise of $\mathbb{N}(0, 0.01)$ and $y_{G3}$ and $y_{G4}$ will have a noise of $\mathbb{N}(0, 0.06)$.
The Extended Kalman filter will be applied to these outputs to better estimate the states in the presence of noise.

%%%%%%%%%%%%%%%%%%%%%%%%%%%%%%%%%%%%%%%%%%%%%%%%%%%%%%%%%%%%%%%%%%%%%%%%%%%%%%%%

\section{PROPOSED METHODS}
TODO: proposed methods
%%%%%%%%%%%%%%%%%%%%%%%%%%%%%%%%%%%%%%%%%%%%%%%%%%%%%%%%%%%%%%%%%%%%%%%%%%%%%%%%

\section{RESULTS}
TODO: results
%%%%%%%%%%%%%%%%%%%%%%%%%%%%%%%%%%%%%%%%%%%%%%%%%%%%%%%%%%%%%%%%%%%%%%%%%%%%%%%%

\section{USING THE TEMPLATE}
\subsection{Some Common Mistakes}
\begin{itemize}
  \item The word "data" is plural, not singular.
\end{itemize}

\subsection{Figures and Tables}

\begin{table}[h]
  \caption{An Example of a Table}
  \label{table_example}
  \begin{center}
    \begin{tabular}{|c||c|}
      \hline
      One   & Two  \\
      \hline
      Three & Four \\
      \hline
    \end{tabular}
  \end{center}
\end{table}


\begin{figure}[thpb]
  \centering
  \framebox{\parbox{3in}{ a picture.}}
  %\includegraphics[scale=1.0]{figurefile}
  \caption{the caption}
  \label{figurelabel}
\end{figure}




% \addtolength{\textheight}{-12cm}   % This command serves to balance the column lengths
% on the last page of the document manually. It shortens
% the textheight of the last page by a suitable amount.
% This command does not take effect until the next page
% so it should come on the page before the last. Make
% sure that you do not shorten the textheight too much.

\section*{APPENDIX}
\subsection*{Nonlinear rotational dynamic terms}
\begin{equation}
  \begin{aligned}
    \Omega_{I_{ez}} & = \dot{\phi}\cos(\epsilon)\cos(\eta) - \dot{\psi}(\sin(\epsilon)\cos(\phi)                                   \\
                    & \quad-\cos(\epsilon)\sin(\eta)\sin(\phi))                                                                    \\
                    & \quad + \dot{\theta}(\sin(\epsilon)\sin(\phi)+\cos(\epsilon)\sin(\eta)\cos(\phi)) - \dot{\eta}\sin(\epsilon)
  \end{aligned}
\end{equation}

\begin{equation}
  \Omega_{dist_{Iy}} = -\dot \phi\sin(\eta) + \dot\theta\cos(\eta)\cos(\phi) + \dot\psi\cos(\eta)\sin(\phi)
\end{equation}

\begin{equation}
  \begin{aligned}
    \Omega_{Iex} = \dot{\epsilon} + \Omega_{dist_{Iy}}
  \end{aligned}
\end{equation}

\begin{equation}
  \begin{aligned}
    \hat{D}^O_{dynUnbz} & = -4.76\alpha_{O_{\text{ex}}} + 0.0006\alpha_{O_{\text{ey}}} + \alpha_{\text{dist\_Ix}}(0.052\cos(x_{G2}) \\
                        & \quad + 14.51\sin(x_{G2})) + \alpha_{\text{dist\_Iy}}(0.364\cos(x_{G2})                                   \\
                        & \quad + 0.052\sin(x_{G2})) + \alpha_{\text{dist\_Iz}}(14.41\cos(x_{G2})                                   \\
                        & \quad + 0.052\sin(x_{G2}))  + 0.07(\Omega_{O_{\text{ex}}})^2                                              \\
                        & \quad - (\Omega_{O_{\text{ey}}}^2)(0.07\cos(x_{G2}) - 0.36\sin(x_{G2}))                                   \\
                        & \quad + 0.3606\Omega_{O_{\text{ex}}}(x_{G3} + \Omega_{dist_{Oz}})                                         \\
                        & \quad - \Omega_{O_{\text{ey}}}\Omega_{O_{\text{ex}}}(-9.235 + 14.41\cos(x_{G2})                           \\
                        & \quad - 0.052\sin(x_{G2}))                                                                                \\
                        & \quad - \Omega_{O_{\text{ey}}}(x_{G3} + \Omega_{dist_{Oz}})(-4.76 + 0.052\cos(x_{G2})                     \\
                        & \quad - 14.41\sin(x_{G2})) - \Omega_{I_{\text{ex}}}x_{G4}(14.51\cos(x_{G2})                               \\
                        & \quad - 0.052\sin(x_{G2}))                                                                                \\
                        & \quad - (x_{G4} + \Omega_{dist{_Iy}})x_{G4}(0.07\cos(x_{G2})                                              \\
                        & \quad - 0.37\sin(x_{G2})) - \Omega_{I_{\text{ez}}}x_{G4}(0.052\cos(x_{G2})                                \\
                        & \quad - 14.41\sin(x_{G2}))
  \end{aligned}
\end{equation}

\begin{equation}
  \begin{aligned}
    \hat{D}^I_{dynUnby} & = 0.067\alpha_{\text{dist\_Ix}} + 0.364\alpha_{\text{dist\_Iz}}                                                                        \\
                        & \quad + \Omega_{I_{\text{ez}}}(14.51\Omega_{I_{\text{ex}}} + 0.067(x_{G4} + \Omega^{\text{dist}}_{Iy} + 0.052\Omega^I_{ez}))           \\
                        & \quad + \Omega_{I_{\text{ex}}}(0.052\Omega_{I_{\text{ex}}} + 0.364(x_{G4} + \Omega^{\text{dist}}_{Iy}) + 14.41\Omega_{I_{\text{ez}}})
  \end{aligned}
\end{equation}

\begin{equation}
  \begin{aligned}
    D^O_{staticUnbz} & = -0.043(-\cos(x_{G1})\sin(w_{G1})       \\  
                     & + \sin(x_{G1})\sin(w_{G2})\cos(w_{G1}))  \\
                     & + 164.8(\sin(x_{G1})\sin(w_{G1})         \\ 
                     & + \cos(x_{G1})\sin(w_{G2})\cos(w_{G1}))
  \end{aligned}
\end{equation}

\begin{equation}
  \begin{aligned}
    D^I_{staticUnby} & = -2.66(\cos(x_{G2})\cos(w_{G1}) - \sin(x_{G2})\sin(w_{G1}))       \\
                     & \quad - 6.44(\cos(x_{G2})\sin(w_{G1}) + \cos(w_{G1})\sin(x_{G2}))
  \end{aligned}
\end{equation}


\subsection*{Angular accelerations}
\begin{equation}
  \begin{aligned}
    \alpha_{O_{ex}} & = \ddot{\phi}\cos(\eta) + \ddot{\theta}\cos(\phi)\sin(\eta) - \ddot{\phi}(\cos(\eta)\sin(\theta)                 \\
                    & \quad  + \cos(\theta)\sin(\eta)\sin(\phi)) + \dot{\eta}(\dot{\phi}\sin(\eta) - \dot{\phi}(\sin(\eta)\sin(\theta) \\
                    & \quad  +  \cos(\eta)\cos(\theta)\sin(\phi)) - \dot{\theta}\cos(\eta)\cos(\phi))                                  \\
                    & \quad + \dot{\phi}\dot{\theta}((\cos(\eta)^2 - \cos(\phi)\sin(\eta)^2)\cos(\theta)                               \\
                    & \quad  + (1 + \cos(\eta))\sin(\eta)\sin(\phi)\sin(\theta))\cos(\phi)                                             \\
                    & \quad + \dot{\phi}(\dot{\theta}(\sin(\phi)\cos(\eta) + \cos(\phi)\sin(\eta))\sin(\phi)                           \\
                    & \quad  - \dot{\psi}(\cos(\phi)\sin(\eta) + \cos(\eta)\sin(\phi))\cos(\phi)\cos(\theta))
  \end{aligned}
\end{equation}

\begin{equation}
  \begin{aligned}
    \alpha_{O_{ey}} & = \ddot{\theta}\cos(\eta)\cos(\phi) - \ddot{\phi}\sin(\eta) + \ddot{\phi}(\sin(\eta)\sin(\theta)                \\
                    & \quad  + \cos(\eta)\cos(\theta)\sin(\phi)) + \dot{\eta}(\dot{\phi}\cos(\eta) + \dot{\theta}\cos(\phi)\sin(\eta) \\
                    & \quad  + \dot{\phi}(\cos(\theta)\sin(\eta)\sin(\phi) - \cos(\eta)\sin(\theta))                                  \\
                    & \quad  + \dot{\phi}\dot{\theta}(\cos(\eta)\cos(\phi)\sin(\phi) - \sin(\eta)  - (\cos(\phi)^2)\sin(\eta)))       \\
                    & \quad + \dot{\phi}\dot{\psi}(\sin(\eta)\sin(\phi) - \cos(\eta)\cos(\phi))\cos(\theta)\cos(\phi)                 \\
                    & \quad + \dot{\phi}\dot{\theta}(\sin(\phi)\sin(\theta)(1 + \cos(\eta))                                           \\
                    & \quad - \cos(\theta)\sin(\eta)(1 + \sin(\eta)))\cos(\eta)\cos(\phi)
  \end{aligned}
\end{equation}

\begin{equation}
  \begin{aligned}
    \alpha_{\text{dist\_Oz}} & = -\ddot{\theta}\sin(\phi) + \ddot{\psi}\cos(\phi)\cos(\theta)                           \\
                             & \quad  + \dot{\psi}\dot{\theta}(\cos(\phi)\cos(\theta)\sin(\eta)\sin(\phi)               \\
                             & \quad  - \sin(\phi)^2\sin(\theta) + \cos(\eta)\cos(\phi)^2\sin(\theta))                  \\
                             & \quad + \dot{\phi}(\dot{\psi}(\cos(\phi)\cos(\theta)\sin(\phi) - \sin(\phi)\sin(\theta)) \\
                             & \quad  + \dot{\theta}\cos(\phi)^2)
  \end{aligned}
\end{equation}

\begin{equation}
  \begin{aligned}
    \alpha_{\text{dist\_Ix}} & = \ddot{\phi}\cos(\epsilon)\cos(\eta) + \ddot{\theta}(\sin(\epsilon)\sin(\phi)                                                                                 \\
                             & \quad  + \cos(\epsilon)\cos(\phi)\sin(\eta)) - \ddot{\psi}(\cos(\epsilon)\cos(\eta)\sin(\theta)                                                                \\
                             & \quad - \cos(\phi)\cos(\theta)\sin(\epsilon)) + \cos(\epsilon)\cos(\theta)\sin(\eta)\sin(\phi)                                                                 \\
                             & \quad  + \dot{\eta}\epsilon_{\dot{}}\cos(\epsilon)  - \dot{\phi}\dot{\theta}\cos(\phi)^2\sin(\epsilon) + \dot{\phi}\dot{\theta}\cos(\epsilon)\cos(\eta)        \\
                             & \quad  + \dot{\eta}\dot{\phi}\cos(\epsilon)\sin(\eta) + \dot\epsilon\dot{\phi}\cos(\eta)\sin(\epsilon) - \epsilon_{\dot{}}\dot{\theta}\cos(\epsilon)\sin(\phi) \\
                             & \quad + \dot{\psi}\dot{\theta}\sin(\epsilon)\sin(\phi)^2\sin(\theta) - \dot{\eta}\dot{\theta}\cos(\epsilon)\cos(\eta)\cos(\phi)                                \\
                             & \quad + \epsilon_{\dot{}}\dot{\psi}\cos(\epsilon)\cos(\phi)\cos(\theta) - \dot\epsilon\dot{\theta}\cos(\epsilon)\sin(\phi)\sin(\eta)                           \\
                             & \quad + \dot{\psi}\dot{\theta}\cos(\epsilon)\cos(\phi)\sin(\phi)\sin(\theta)                                                                                   \\
                             & \quad - \dot{\phi}\dot{\theta}\cos(\epsilon)\cos(\eta)\cos(\phi)^2                                                                                             \\
                             & \quad - \dot{\psi}\dot{\theta}\cos(\epsilon)\cos(\phi)^2\cos(\theta)\sin(\eta)^2                                                                               \\
                             & \quad - \dot{\eta}\dot{\psi}\cos(\epsilon)\cos(\eta)\cos(\theta)\sin(\phi)                                                                                     \\
                             & \quad + \dot{\phi}\dot{\theta}\cos(\epsilon)\cos(\phi)\sin(\eta)\sin(\phi)                                                                                     \\
                             & \quad - \dot{\phi}\dot{\psi}\cos(\phi)\cos(\theta)\sin(\epsilon)\sin(\phi)                                                                                     \\
                             & \quad + \epsilon_{\dot{}}\dot{\psi}\cos(\theta)\sin(\epsilon)\sin(\eta)\sin(\phi)
  \end{aligned}
\end{equation}

\begin{equation}
  \begin{aligned}
    \alpha_{\text{dist\_Iy}} & = -\ddot{\phi}\sin(\eta) + \ddot{\psi}(\cos(\eta)\cos(\theta)\sin(\phi) + \sin(\eta)\sin(\theta))                         \\
                             & \quad + \ddot{\theta}\cos(\eta)\cos(\phi) - \dot{\phi}\dot{\theta}\sin(\eta) + \dot{\eta}\dot{\phi}\cos(\eta)             \\
                             & \quad + \dot{\phi}\dot{\theta}\cos(\phi)^2\sin(\eta) + \dot{\psi}\dot{\theta}\cos(\eta)^2\cos(\phi)\sin(\phi)\sin(\theta) \\
                             & \quad + \dot{\eta}\dot{\theta}\cos(\phi)\sin(\eta) - \dot{\eta}\dot{\psi}\cos(\eta)\sin(\theta)                           \\
                             & \quad + \dot{\phi}\dot{\theta}\cos(\eta)\cos(\phi)\sin(\phi) + \dot{\eta}(\dot{\psi}\cos(\theta)\sin(\eta)\sin(\phi)      \\
                             & \quad - \dot{\phi}\dot{\psi}\cos(\eta)\cos(\phi)^2 \cos(\theta)                                                           \\
                             & \quad - \dot{\psi}\dot{\theta}\cos(\eta)\cos(\phi)\cos(\theta)\sin(\eta)                                                  \\
                             & \quad + \dot{\phi}\dot{\psi}\cos(\phi)\cos(\theta)\sin(\eta)\sin(\phi)                                                    \\
                             & \quad + \dot{\psi}\dot{\theta}\cos(\eta)\cos(\phi)\sin(\phi)\sin(\theta))                                                 \\
                             & \quad - \dot{\psi}\dot{\theta}\cos(\eta)\cos(\phi)^2\cos(\theta)\sin(\eta)
  \end{aligned}
\end{equation}

\begin{equation}
  \begin{aligned}
    \alpha_{\text{dist\_Iz}} & = \ddot{\phi}\cos(\eta)\sin(\epsilon) + \ddot{\theta}(\cos(\phi)\sin(\epsilon)\sin(\eta)                                \\
                             & \quad  - \cos(\epsilon)\sin(\phi)) - \ddot{\psi}(\cos(\epsilon)\cos(\phi)\cos(\theta)                                   \\
                             & \quad - \cos(\eta)\sin(\epsilon)\sin(\theta) + \cos(\theta)\sin(\epsilon)\sin(\eta)\sin(\phi))                          \\
                             & \quad + \dot{\eta}(\epsilon_{\dot{}} + \dot{\phi}\sin(\eta) - \dot{\psi}\sin(\eta)\sin(\theta)                          \\
                             & \quad - \dot{\psi}\cos(\eta)\cos(\theta)\sin(\phi) - \dot{\theta}\cos(\eta)\cos(\phi))\sin(\epsilon)                    \\
                             & \quad  + \dot\epsilon_{\dot{}}\dot{\psi}(\cos(\epsilon)\cos(\eta)\sin(\theta) + \cos(\phi)\cos(\theta)\sin(\epsilon)    \\
                             & \quad - \cos(\epsilon)\cos(\theta)\sin(\eta)\sin(\phi)) - \dot\epsilon\dot{\phi}(\cos(\epsilon)\cos(\eta)               \\
                             & \quad - \dot\epsilon_{\dot{}}\dot{\theta}(\sin(\epsilon)\sin(\phi) - \cos(\epsilon)\cos(\phi)\sin(\eta)))               \\
                             & \quad  + \dot{\phi}\dot{\theta}(\cos(\phi)\sin(\epsilon)\sin(\epsilon)\sin(\eta)\sin(\phi) + \cos(\epsilon)\cos(\phi)^2 \\
                             & \quad + \cos(\eta)\sin(\epsilon) - \cos(\eta)\cos(\phi)^2\sin(\epsilon))                                                \\
                             & \quad  + \dot{\psi}\dot{\theta}(\cos(\epsilon)\cos(\eta)\cos(\phi)^2\sin(\theta)                                        \\
                             & \quad  + \cos(\eta)^2\cos(\phi)\cos(\theta)\sin(\epsilon)                                                               \\
                             & \quad - \cos(\epsilon)\sin(\phi)^2\sin(\theta)) - \cos(\phi)^2\cos(\theta)\sin(\epsilon)\sin(\eta)^2                    \\
                             & \quad  + \cos(\epsilon)\cos(\phi)\cos(\theta)\sin(\eta)\sin(\phi)                                                       \\
                             & \quad + \cos(\eta)\cos(\phi)\sin(\epsilon)\sin(\eta)\sin(\phi)\sin(\theta)                                              \\
                             & \quad + \dot{\phi}\dot{\psi}(\cos(\epsilon)\cos(\phi)\cos(\theta)\sin(\phi)                                             \\
                             & \quad  - \cos(\epsilon)\sin(\phi)\sin(\theta) - \cos(\phi)^2\cos(\theta)\sin(\epsilon)\sin(\eta))                       \\
                             & \quad - \cos(\eta)\cos(\phi)\cos(\theta)\sin(\epsilon)\sin(\phi)
  \end{aligned}
\end{equation}


\begin{thebibliography}{99}
  \bibitem{c1} Erhan Poyrazoglu, "Detailed Modeling and Control of a 2-DOF Gimbal System", Master's Thesis, Dept. Elect. and Electronics Eng., Middle East Technical University, 2017.
  % \bibitem{c1} G. O. Young, "Synthetic structure of industrial plastics (Book style with paper title and editor)," 	in Plastics, 2nd ed. vol. 3, J. Peters, Ed.  New York: McGraw-Hill, 1964, pp. 15"64.
  % \bibitem{c2} W.-K. Chen, Linear Networks and Systems (Book style).	Belmont, CA: Wadsworth, 1993, pp. 123"135.
  % \bibitem{c4} B. Smith, "An approach to graphs of linear forms (Unpublished work style)," unpublished.
  % \bibitem{c10} J. U. Duncombe, "Infrared navigation"Part I: An assessment of feasibility (Periodical style)," IEEE Trans. Electron Devices, vol. ED-11, pp. 34"39, Jan. 1959.
  % \bibitem{c13} S. P. Bingulac, "On the compatibility of adaptive controllers (Published Conference Proceedings style)," in Proc. 4th Annu. Allerton Conf. Circuits and Systems Theory, New York, 1994, pp. 8"16.
  % \bibitem{c18} J. Williams, "Narrow-band analyzer (Thesis or Dissertation style)," Ph.D. dissertation, Dept. Elect. Eng., Harvard Univ., Cambridge, MA, 1993. 
  
  
  
  
  
  
\end{thebibliography}




\end{document}
